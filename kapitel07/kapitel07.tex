\chapter{Conclusion}
\label{chap:Con}
%\vspace{0.4cm}

In a nutshell, it is easy to recognize that an extensive amount of literature has already been involved with cognitive radio\footnote{For instance, as on 01.05.16, 17915 search results are retrieved upon typing the keyword cognitive radio in IEEE Xplore, a database for scientific publications available at \url{http://ieeexplore.ieee.org/Xplore/home.jsp}.}. Despite its huge popularity and in-depth knowledge acquired on this topic, an autonomous as well as exhaustive implementation of such a concept is underdeveloped. One main reason behind this is the fact that the existing models (developed for the performance characterization) have focused more on theoretical analysis and little on the hardware deployment. In this regard, due to the complexity of the underlying problem, these models tend to overlook certain aspects such as imperfections arising from the analog front end or channel knowledge that are fundamental for the hardware implementation. 

The lack of these imperfections in the system model renders the performance analysis of the CR system incomplete, thereby prohibiting its hardware implementation. The knowledge of the involved channels between the primary and the secondary systems is one of such aspects dealt in this thesis. 
 %Moreover, at several instances in the thesis, it has been consistently argued that the channels' knowledge is principle to the CR systems. 
From a physical layer perspective, it has been identified that the channel knowledge is extremely necessary for the realization of the CR techniques on a hardware, thus, allowing a CR system to control the interference accumulated by the primary system. In this thesis, this notion has been extensively justified and resolved through adequate analysis while considering a hardware deployment. %As a matter of concern, an extensive investigation in this direction is still lacking in the literature. 

%Motivated by this fact, this thesis incorporate channel estimation that facilitate hardware deployment. 
Above all, the inclusion of the channel estimation requires a proper allocation of time resources in the frame structure and appropriate measures to counter variations due to the estimation error induced in the system. Surely, these factors have a detrimental effect on the performance of a CR system, leading to the performance degradation. These channel estimation related issues have been carefully identified and characterized in this thesis, which ultimately allows us to depict the performance of the CR systems in a fairly realistic scenario. Besides, following the deployment perspective, a received power-based channel estimation technique is proposed for the estimation, particularly, for the channels that exist between the two systems. 

Briefly, the analysis performed in the thesis does not only provide answers to specific questions related to imperfect channel knowledge, including 
\begin{enumerate} \item How to counter the uncertain interference induced in different CR systems? \item How to evaluate the amount of performance degradation? \item How to determine the suitable estimation time and suitable sensing time that yields the maximum throughput achieved? \end{enumerate}
but also give emphasis to techniques such as \begin{enumerate} \item Implementation of the channel estimation at the secondary system \item Energy-based detection and \item Received power-based channel estimation, \end{enumerate} that ultimately encourage hardware implementation of CR systems. 


\section{Summary}
%\todo{This section needs reconsideration}
This section summarizes the major findings from each chapter. 

Chapter \ref{chap:IS} outlined the fact that the spectrum sensing mechanism can be accomplished at the ST only if the knowledge of the involved channels (namely, sensing, interference and access channels) is accessible at the ST. Through analysis, it is clearly identified that the channel estimation degrades the performance of the IS. In this context, an estimation-sensing-throughput tradeoff is established that allows us to regulate this performance degradation, and consequently determine the achievable throughput at the SR.

On the similar basis, Chapter \ref{chap:US} showed that the power control mechanism requires the knowledge of the involved channels (namely, interference -- primary and secondary -- channel and access channel) at the ST. The estimation-throughput tradeoff, a novel approach to jointly characterize the performance of the US and analyze the performance degradation, is presented. Following the main analysis, it is indicated that the performance degradation of the US can be effectively controlled only if the estimation time is selected appropriately.   

The notion of the channel estimation presented in the previous two chapters is extended to the hybrid scenario in Chapter \ref{chap:HS}, whereby the spectrum sensing and the power control mechanisms are simultaneously enabled at the ST. Through numerical analysis, it is emphasized that a significant performance gain is achievable by combining the IS and the US. Similar to Chapter \ref{chap:IS}, an estimation-sensing-throughput tradeoff is formulated that allowed us to investigate the variation of the achievable throughput along the estimation and the sensing time. 

While the previous three chapters focused on the theoretical analysis, the hardware feasibility in terms of the validation and the demonstration of the proposed analysis are discussed in Chapter \ref{chap:HVD}. In this regard, a software defined radio platform following the guidelines of an US is considered for the deployment. 


\section{Outlook}
%\todo{Lets talk about the major drawbacks of the thesis and futuristic research works.} 
With regard to the performance analysis and the deployment-centric viewpoint towards CR systems emphasized in the thesis, the following extensions or considerations to the proposed framework could be of great interest for future investigations. For instance, this thesis focused only on a half duplex CR communication, i.e., the CR techniques (which include spectrum sensing and power control) are time-interlaced with the data transmission. Recently, there has been significant advancement concerning the feasibility of in-band full duplex communication, for a detailed discussion over in-band full duplex communication, please consider \cite{Bhar13, Sab14, Liu15} and the references therein. In this context, the CR communication can be transformed into the full duplex in-band, whereby the CR techniques and the data transmission occurs simultaneously in time and over the same frequency channel. The design challenges and the corresponding performance tradeoffs related to the in-band full duplex CR communication are precisely dealt in \cite{Liao15, Kim15}. 

In addition, the performed analysis considers that the ST and the SR are installed with single antenna. Considering the fact that the state-of-the-art standards are mostly equipped with multiple antennas, in order to corroborate the preliminary analysis involving channel estimation, the performance improvement procured by upgrading the existing spectrum sensing (detector performance) due to the deployment of multiple antennas \cite{Dig07,Tah10} has been completely neglected in the thesis. Through hardware deployment, the authors in \citeK{Kaushik16_VTC2} argued that the hardware complexity in context with the CR system escalates with the deployment of multiple antennas, prohibiting the usage of well-known combining techniques such as equal-gain combining and maximum-ratio combining \cite{Alouini03}. In this regard, non-conventional techniques, such as square-law selector and square-law combiner (following the principle of energy detection) are able to reduce complexity, thereby promoting the feasibility of multiple antennas. 

Given the complexity of the underlying problem, impairments due to the asynchronous (in time domain) access by the secondary system to the licensed spectrum is left aside throughout the thesis. The asynchronous access is due to the unknown (which can be random also) behaviour of PU traffic. In these circumstances, the assumption concerning the synchronous access (perfect alignment to the primary systems' medium access) becomes invalid. As a consequence, this asynchronous access certainly has an impact on the performance of the CR systems. A careful integration of the asynchronous access to the proposed analysis presents a promising research direction. To tackle this problem, the reader is encouraged to consult the following references \cite{Jiang13_, Jiang15}.

Also, the performance evaluation presented in the thesis considers symmetric fading, i.e., the channel gains are subjected to the same value of $m$ (which represents Nakagami-$m$ parameter, characterizing the severity of fading in the channel). However, depending on the deployment scenario, the derived expressions can be utilized to realize asymmetric fading \cite{Sura08} by substituting different values of $m$ corresponding to different channels. In this regard, the proposed framework can be extended to study the influence of asymmetric fading on the performance. 

Finally, the performance analysis in this thesis has been confined to PT, PR and ST and SR as the main entities, a classical approach of describing a CR system. The effect of the presence of other PTs and other STs in the network on the performance, described in terms of different performance parameters such as spatial interference at the PR and spatial throughput at the SR has not been treated in the thesis. The concept of stochastic geometry, widely accepted for modelling the wireless networks, has been recently applied to the perform analysis for the cognitive radio networks. In order to establish an in-depth understanding of this concept, it is advisable to consult the following references \cite{Ghasemi08, Lee12, Kusal12, Kusal13, Elsawy13, Song14}, \citeK{Kaushik14_P}. 

As a closing remark, spectrum is a precious component that enables wireless connectivity to the billions of devices residing within the 5G network. To meet this escalating demand for spectrum, cognitive radio, competing with technologies such as the mm-Wave technology and the visible light communication, represents a viable option. Having said that, there exist certain scenarios, including the one considered in the thesis (refer Chapter \ref{chap:Int}), that facilitate the co-existence to these technologies. 







