\chapter{Conclusion}
\label{chap:Con}

Looking backwards, one can easily figure out that an extensive amount of literature has already been dedicated to the cognitive radio\footnote{For instance, as on 01.05.16, 17915 search results are retrieved upon typing the keyword cognitive radio in IEEE Xplore, a database for scientific publications available at \url{http://ieeexplore.ieee.org/Xplore/home.jsp}.}. Despite its huge popularity and in-depth knowledge acquired on this topic, an autonomous as-well-as exhaustive implementation of such a concept is underdeveloped. One main reason behind this is the fact that the existing models (developed for the performance characterization) have laid little emphasis on the hardware deployment. In this regard, due to the complexity of the problem, these models tend to overlook certain aspects, such as imperfections arising from the analog front end or channel, which otherwise can be critical for the deployment perspective, thereby prohibiting its hardware implementation. The knowledge of the involved channels between the primary and the secondary systems is one of such aspects dealt in this thesis. 
 %Moreover, at several instances in the thesis, it has been consistently argued that the channels' knowledge is principle to the CR systems. 
From a physical layer perspective, it has been identified that the channel knowledge is extremely necessary for the realization of the CR techniques on a hardware, thus, allows a CR system to control the interference accumulated by the primary system. This notion has been carefully argued and justified through numerical analysis in the thesis. %As a matter of concern, an extensive investigation in this direction is still lacking in the literature. 

%Motivated by this fact, this thesis incorporate channel estimation that facilitate hardware deployment. 
Above all, the inclusion of the channel estimation requires a proper allocation of time resources in the frame structure and techniques to deal variations due to the estimation error induced the system. Surely, these factors have a detrimental effect on the performance of a CR system, leading to performance degradation. These channel estimation related issues have been identified and characterized in this thesis that ultimately allows us to depict the performance of the CR systems in a fairly realistic scenario. Besides, following the deployment perspective, a received power based channel estimation technique is proposed for the estimation of the channels between the two systems.  


\section{Summary}




\section{Outlook}










