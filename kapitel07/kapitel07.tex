\chapter{Conclusion}
\label{chap:Con}

Looking backwards, one can easily figure out that an extensive amount of literature has already been dedicated to the cognitive radio\footnote{For instance, as on 01.05.16, 17915 search results are retrieved upon typing the keyword cognitive radio in IEEE Xplore, a database for scientific publications available at \url{http://ieeexplore.ieee.org/Xplore/home.jsp}.}. Despite its huge popularity and in-depth knowledge acquired on this topic, an autonomous as-well-as exhaustive implementation of such a concept is underdeveloped. One main reason behind this is the fact that the existing models (developed for the performance characterization) have laid little emphasis on the hardware deployment. In this regard, due to the complexity of the problem, these models tend to overlook certain aspects, such as imperfections arising from the analog front end or channel, which render the performance analysis of the CR system incomplete, thereby prohibiting its hardware implementation. The knowledge of the involved channels between the primary and the secondary systems is one of such aspects dealt in this thesis. 
 %Moreover, at several instances in the thesis, it has been consistently argued that the channels' knowledge is principle to the CR systems. 
From a physical layer perspective, it has been identified that the channel knowledge is extremely necessary for the realization of the CR techniques on a hardware, thus, allows a CR system to control the interference accumulated by the primary system. In this thesis, this notion has been carefully argued and justified through numerical analysis. %As a matter of concern, an extensive investigation in this direction is still lacking in the literature. 

%Motivated by this fact, this thesis incorporate channel estimation that facilitate hardware deployment. 
Above all, the inclusion of the channel estimation requires a proper allocation of time resources in the frame structure and appropriate measures to counter variations due to the estimation error induced the system. Surely, these factors have a detrimental effect on the performance of a CR system, leading to performance degradation. These channel estimation related issues have been carefully identified and characterized in this thesis, which ultimately allows us to depict the performance of the CR systems in a fairly realistic scenario. Besides, following the deployment perspective, a received power based channel estimation technique is proposed for the estimation particularly for the channels that exists between the two systems. Finally, the theoretical analysis performed in the thesis not only allow us to counter the excessive interference induced in the system but also helps us evaluate the suitable estimation time and suitable sensing time that yields the maximum throughput achieved by the secondary system. 


\section{Summary}

This section enlist the main findings from each chapter. 

The chapter \ref{chap:IS} justified that the spectrum sensing mechanism can be performed by the ST only if the knowledge of the involved channels (namely, sensing, interference and access channels) is accessible at the ST. It is well indicated that channel estimation degrades the performance of the IS. An estimation-sensing-throughput tradeoff is established to examine the performance degradation due to the channel estimation.

On similar basis, chapter \ref{chap:US} augmented that the power control mechanism requires the knowledge of the involved channels (namely, interference -- primary and secondary -- and access channel) at the ST. An estimation-throughput tradeoff presents a significant approach of analyzing the performance degradation. Following the main analysis, it is indicated that the performance degradation of the US can be effectively controlled only if the estimation time is selected appropriately.   

Chapter \ref{chap:HS} extends the proposal of the channel estimation presented in the previous two chapters to the hybrid scenario, whereby the spectrum sensing and the power control mechanisms are enabled at the ST. Through numerical analysis, it is justified that the significant performance gain is achievable by combing the IS and the US. Similar to chapter \ref{chap:IS}, an estimation-sensing-throughput tradeoff is obtained that allows us to investigate the variation of the achievable throughput along the estimation and the sensing time. 

Chapter \ref{chap:HVD}


\section{Outlook}










