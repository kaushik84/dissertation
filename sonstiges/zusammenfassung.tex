\chapter*{Summary}
%Der Grundgedanke bei Software Defined Radios besteht darin, Funkstandards in Software auf rekonfigurierbaren Prozessoren zu verarbeiten. Dieses Konzept f�r neue Funkger�te existiert bereits seit mehreren Jahrzehnten. Allerdings ist es erst durch die rasante Entwicklung auf dem Markt der Prozessoren der letzten Jahre m�glich, SDR-Plattformen kommerziell zu erwerben und eigene Wellenformen zu implementieren. Dabei ergibt sich das Problem, dass Wellenformen, die auf der eigenen Plattform lauff�hig sind, zu anderen Plattform nicht zwangsl�ufig kompatibel sind. Dieses Problem tritt  auch bei einer m�glichen Platt\-formerneuerung auf. Die Wellenformen, die f�r eine alte Plattform entwickelt wurden, m�ssen auf eine neue Plattform portiert werden. Dabei stellt sich zwangsl�ufig die Frage: ,,Wie k�nnen Wellenformen entwickelt werden, die auf beliebigen Plattformen lauff�hig sind?''

%Ein Erfolg versprechender Ansatz ist die Entwicklung von Wellenformen nach der Model Driven Architecture. Diese beschreibt einen Entwicklungsprozess, der sich von sehr generischen, plattformunabh�ngigen Modellierungen der Funktionalit�t bis zum ausf�hrbaren Bin�rcode erstreckt. In dieser Arbeit wird vorgestellt, wie dieser Prozess auf die portable Entwicklung von Wellenformen angepasst werden kann. Die automatisierte Erzeugung von Quellcode spielt hierbei eine wichtige Rolle. Daher werden Laufzeit- und Speichermessungen vorgestellt, die generierten Code mit nicht optimiertem und optimiertem Code vergleichen  und damit einen Einblick in die Effizienz von automatisch generiertem Code erlauben.

%Das Ettus USRP und das Lyrtech Small Form Factor SDR geh�ren zu den kommerziell erfolgreichsten SDR-Plattformen. Daher werden in dieser Arbeit ausf�hrlich ihr Aufbau sowie ihre F�higkeiten und Limitierungen beschrieben. Weiterhin wird aufgezeit, wie diese Plattformen in den be\-reits vorgestellten Entwicklungsprozess integriert werden k�nnen. Dazu geh�ren sowohl die Umsetzung der programmierbaren Schnittstellen und der Bussysteme in die Systemmodellierung als auch die Integration der betreffenden Prozessoren in die Codegenerierung.

%Um zu demonstrieren, dass der vorgestellte Entwicklungsprozess auch praktisch anwendbar ist, wurde die Wellenform TETRA f�r eine Platt\-form entwickelt und auf eine zweite Plattform portiert. Die Umsetzung der dabei zu realisierenden Modelle werden in dieser Arbeit genauso vorgestellt wie die Verarbeitungsdauer der Algorithmen auf den betreffenden Prozessoren. Um zu gew�hrleisten, dass die Wellenform standardkonform umgesetzt wurde, kamen Messger�te zum Einsatz, die sowohl Sende- als auch Empfangspfad gegen die TETRA-Spezifikation getestet haben. Neben TETRA wurden zwei weitere Wellenformen ent\-wickelt und portiert. Die Ergebnisse und Herausforderungen all dieser Entwicklungsvorg�nge werden in dieser Arbeit pr�sentiert.
   

  

\cleardoublepage

\chapter*{Abstract}

%The basic idea of Software Defined Radio is the implementation of radio communication standards with software on reconfigurable processors. This concept for new radio devices was already proposed several decades ago. However, through the rapid development of processors in recent years, it is possible to acquire commercial SDR platforms and build own waveforms. Unfortunately, waveforms that are developed for one platform are not necessarily compatible to other platforms. This problem also occurs with a possible hardware upgrade. The waveforms that were developed for an old platform must be ported to a new platform. Therefore, this work focuses on the question: ``How can we build waveforms that can be moved from one platform to another?''

%A promising approach is the development of waveforms based on the Model Driven Architecture. It describes a development process that extends from a very generic, platform-independent functionality to the executable binary code. This work presents the adaption of this process to the development of portable waveforms. In this process, the automated generation of source code plays a decisive role. Therefore, measurements of processing time and memory consumption are presented to compare the generated code with non-optimized and optimized code and allow insights into the efficiency of automatically generated code.

%The USRP from Ettus and the Small Form Factor SDR from Lyrtech are among the most used commercially SDR platforms. Therefore, their structure as well as their capabilities and limitations are presented in this work. It is furthermore shown, how these platforms can be integrated into the development flow. This includes the implementation of programmable interfaces and bus systems as well as the integration of the processors in the code generation process.

%To demonstrate that the used development flow is also applicable in practice, a proof of concept is given with the development and port of a TETRA waveform from one platform to another. Therefore, the realizations of the waveform for both platforms are presented as well as the processing times for the algorithms on the different processors. To demonstrate the standard compliance, the waveform was tested with measurement equipment against the TETRA air specification. In addition to TETRA, two other waveforms were developed and ported. This work presents the results and challenges of all these waveform developments and ports.


\cleardoublepage
