\newcommand{\FBTableHead}{%
  \multicolumn{2}{@{}c@{}}%
  {\centering \bf Forschungsberichte aus dem Institut f�r Nachrichtentechnik}\\
  \multicolumn{2}{@{}c@{}}{\bf des Karlsruher Instituts f�r
    Technologie}\\ 
    \multicolumn{2}{@{}c}{Herausgeber: Univ.-Prof. Dr.\,rer.\,nat. Friedrich Jondral}\\
\rule{0pt}{3ex}  
  \\}
%
\begin{tabular}{@{}p{1.4cm}p{9.2cm}@{}}
  \FBTableHead%
Band 1&Marcel Kohl\\
  &{\bf Simulationsmodelle f�r die Bewertung von \newline Satelliten�bertragungsstrecken im
    \newline 20/30 GHz Bereich}\\
\rule{0pt}{3ex}%
Band 2&Christoph Delfs\\
  &\textbf{Zeit-Frequenz-Signalanalyse: Lineare und \newline quadratische
    Verfahren sowie vergleichende \newline 
    Untersuchungen zur Klassifikation von Klaviert�nen}\\%
\rule{0pt}{3ex}%
Band 3&Gunnar Wetzker\\%
  &{\bf Maximum-Likelihood Akquisition von Direct\newline Sequence Spread-Spectrum Signalen}\\
\rule{0pt}{3ex}%
Band 4&Anne Wiesler\\
  &{\bf Parametergesteuertes Software Radio \newline f�r Mobilfunksysteme}\\%
\rule{0pt}{3ex}%
Band 5&Karl L�tjen\\
  &{\bf Systeme und Verfahren f�r strukturelle \newline Musteranalysen
    mit Produktionsnetzen}\\\rule{0pt}{3ex}%
Band 6&Ralf Machauer\\
  &{\bf Multicode-Detektion im UMTS}\\\rule{0pt}{3ex}%
Band 7&Gunther M. A. Sessler\\
  &{\bf Schnell konvergierender Polynomial Expansion \newline Multiuser  Detektor mit niedriger Komplexit�t}\\
\rule{0pt}{3ex}%
Band 8&Henrik Schober\\
  &{\bf Breitbandige OFDM Funk�bertragung bei\newline hohen
    Teilnehmergeschwindigkeiten}\\\rule{0pt}{3ex}% 
Band 9&Arnd-Ragnar Rhiemeier\\
  &{\bf Modulares Software Defined Radio}\\\rule{0pt}{3ex}%
Band 10 & Mustafa Meng\"{u}\c{c} �ner\\
&{\bf Air Interface Identification for \newline Software Radio Systems}\\
\rule{0pt}{3ex}
\end{tabular}

\newpage
\begin{tabular}{@{}p{1.4cm}p{9.2cm}@{}}
 \FBTableHead
Band 11 & Fatih \c{C}apar\\
&{\bf Dynamische Spektrumverwaltung und  \newline elektronische
  Echtzeitvermarktung von \newline
Funkspektren in Hotspotnetzen}\\%
\rule{0pt}{3ex}%
Band 12 & Ihan Martoyo\\
&{\bf Frequency Domain Equalization in CDMA Detection}\\%
\rule{0pt}{3ex}%
Band 13 & Timo Wei\ss\\
&{\bf OFDM-basiertes Spectrum Pooling}\\%
\rule{0pt}{3ex}%
Band 14 & Wojciech Kuropatwi\'{n}ski-Kaiser\\
&{\bf MIMO-Demonstrator basierend \newline auf GSM-Komponenten}\\%
\rule{0pt}{3ex}%
Band 15 & Piotr Rykaczewski\\
&{\bf Quadraturempf�nger f�r Software Defined Radios: \newline Kompensation von Gleichlauffehlern}\\%
\rule{0pt}{3ex}%
Band 16 & Michael Eisenacher\\
&{\bf Optimierung von Ultra-Wideband-Signalen (UWB)}\hfill\\%
\rule{0pt}{3ex}%
Band 17 & Clemens Kl�ck\\
&{\bf Auction-based Medium Access Control}\\%
\rule{0pt}{3ex}%
Band 18 & Martin Henkel\\
&{\bf Architektur eines DRM-Empf�ngers \newline und
  Basisbandalgorithmen zur Frequenzakquisition \newline  und Kanalsch�tzung}\\%
\rule{0pt}{3ex}%
Band 19 & Stefan Edinger\\
&{\bf Mehrtr�gerverfahren mit dynamisch-adaptiver \newline Modulation
  zur unterbrechungsfreien \newline Daten�bertragung  in
  St�rf�llen}\\%
\rule{0pt}{3ex}%
Band 20 & Volker Blaschke\\
&{\bf Multiband Cognitive Radio-Systeme}\\
&\\\end{tabular}

\begin{tabular}{@{}p{1.4cm}p{9.2cm}@{}}
  \FBTableHead%
\rule{0pt}{3ex}%
Band 21 & Ulrich Berthold\\
&{\bf Dynamic Spectrum Access using OFDM-based \newline Overlay Systems}\\%
\rule{0pt}{3ex}%
Band 22 & Sinja Brandes\\
&{\bf Suppression of Mutual Interference in \newline OFDM-based Overlay  Systems}\\%
\rule{0pt}{3ex}%
Band 23 & Christian K{\"o}rner\\
&{\bf Cognitive Radio -- Kanalsegmentierung und  \newline Sch{\"a}tzung von Periodizit{\"a}ten}\\%
\rule{0pt}{3ex}%
Band 24 & Tobias Renk\\
&{\bf Cooperative Communications: Network Design and \newline Incremental Relaying}\\%
\rule{0pt}{3ex}%
Band 25 & Dennis Burgkhardt\\
&{\bf Dynamische Reallokation von spektralen Ressourcen \newline in
  einem hierarchischen Auktionssystem}\\
\rule{0pt}{3ex}%
Band 26 & Stefan Nagel\\
&{\bf Portable Waveform Development for \newline Software Defined Radios}\\
&\\\end{tabular} 
