\chapter*{Vorwort des Herausgebers}

%Software Defined Radios (SDRs) werden inzwischen seit zwei Jahrzehnten intensiv untersucht. Dabei stand zun�chst die Frage im Vordergrund, wie ein Funkger�t entwickelt werden kann, das mehrere verschiedene Standards unterst�tzt. Danach wurden in weitergehenden Untersuchungen Verfahren zum Software-Update, auch durch Funk�bertragung, implementiert. Dazu geh�rten auch Methoden zum Nachweis der Korrektheit der �bertragenen Software. Breitfl�chig durchgesetzt haben sich die SDR Technologien bisher im Wesentlichen in Basisstationsger�ten. Mobile Endger�te existieren vor allen Dingen als Multiband- (GSM 900, 1800, 1900) und Multistandard- (GSM, UMTS) Transceiver.

%Eine weitere wesentliche Eigenschaft, durch die sich SDRs auszeichnen k�nnen, wurde bisher zwar in der Literatur breit diskutiert, experimentell jedoch, wohl wegen der dazu notwendigen Hardware, nicht angegangen. Es handelt sich um die \textbf{Portabilit�t} von Wellenformen\footnote{Die Begriffe Standard und Wellenform werden oft als synonym angesehen. Allerdings umfassen Wellenformen, im Gegensatz zu Standards, h�ufig neben der Definition einer Luftschnittstelle auch Eigenschaften des Empf�ngers.} von einer Hardwareplattform auf eine andere. Durch das Aufkommen finanzierbarer Plattformen wie der Universal Software Radio Peripheral (USRP) von Ettus Research oder der Small Form Factor Software Defined Radio Development Platform (SFF SDR DP) von Lyrtech wird eine experimentelle Untersuchung der Portabilit�t von Wellenformen durchf�hrbar. Dabei treten Fragen nach einer Entwicklungsstrategie f�r die SDR Software unter Ber�cksichtigung der Portabilit�t sowie nach den Kosten, die eine Portierung verursacht, in den Vordergrund. Um hierf�r Antworten finden zu k�nnen, m�ssen, neben einem profunden Wissen �ber die beteiligten SDR Plattformen, Kenntnisse �ber Wellenformen bzw. Standards vorliegen. Dazu kommt die Beherrschung der notwendigen Hilfsmittel wie Programmiersprachen von MatLab �ber Python und C bis hin zu VHDL, Verilog und Assembler sowie der Umgang mit Messmitteln wie Signalgeneratoren und -analysatoren, die f�r den Nachweis erfolgreicher Arbeit, der die Interoperabilit�t zwischen auf verschiedenen Hardwareplattformen basierenden Funkger�ten umfasst, gebraucht werden.

A%uf der Basis des am KIT Institut f�r Nachrichtentechnik (Communications Engineering Lab, CEL) eingerichteten Funklabors hat Stefan Nagel die Portabilit�t verschiedener Wellenformen zwischen den beiden oben genannten Hardwareplattformen USRP und SFF SDR DP untersucht. Mit der Dissertation \emph{Portable Waveform Development for Software Defined Radios} legt er die wesentlichen Ergebnisse seiner Arbeit vor.

%Die Arbeit tr�gt insofern nachhaltig zum Fortschritt von Wissenschaft und Technik bei, als hier erstmalig, basierend auf der Model Driven Architecture (MDA) der Object Management Group (OMG), die Software f�r die Signalverarbeitung verschiedener Funk�bertragungsstandards entwickelt, implementiert und von einer Plattform auf eine andere portiert wurde. 


Karlsruhe, im 20XX\newline
Friedrich Jondral




