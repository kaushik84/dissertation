
\chapter*{Abstract}
Over the last decade, wireless communication is witnessing a tremendous growth in the data traffic due to ever-increasing number of connected devices. Certainly, in future, the state-of-the-art standards are incapable of sustaining the substantial amount of data traffic, originating from these devices. It is being visualized that a major portion of this requirement can be satisfied through an additional spectrum. Due to exclusive usage, the spectrum below $\SI{6}{GHz}$ is not able to meet this demand of additional spectrum, leading to its scarcity. To this end, Cognitive Radio (CR), along with millimeter-wave technology and visible-light communication, is envisaged as an alternative source of spectrum. The latter techniques are limited to a point-to-point communication, by which mobility is compromised. In contrast, a CR system aims at an efficient utilization of the spectrum below $\SI{6}{GHz}$ -- suitable for mobile communications -- by enabling a secondary access to the licensed spectrum while ensuring a sufficient protection to the licensed users (also referred as a primary system). %Due to this strict constraint on the interference caused by the shared access, the performance analysis has been critical to a cognitive radio system that, in-turn, encourages its hardware deployment. 


Despite the fact that an extensive amount of literature has been dedicated to the field of CR, its performance analysis has been dealt inadequately from a deployment perspective. Therefore, making it difficult to understand the extent of vulnerability caused to the primary system. Motivated by this fact, this thesis establishes a deployment-centric viewpoint for analyzing the performance of a CR system. Following this viewpoint, it has been identified that the involved channels' knowledge at the secondary transmitter is pivotal for the realization of cognitive techniques. However, the aspect of channel knowledge in context to CR systems, particularly its effect on the performance, has not been clearly understood. With the purpose of curtailing this gap, this thesis proposes a successful integration of this knowledge -- by carrying out channel estimation -- in reference to different CR systems, namely interweave, underlay and hybrid systems. More specifically, the thesis outlines the following three aspects corresponding to the aforementioned CR systems, employing different cognitive techniques such as spectrum sensing, power control and their combination.

First, this thesis establishes an analytical framework to characterize the effects such as time allocation and variation, arising due to the incorporation of imperfect channel knowledge, that are detrimental to the performance of the CR systems. In order to facilitate hardware deployment of a CR system, received power-based estimation, a novel channel estimation technique is employed for the channels existing between the primary and the secondary systems, thus fulfilling low-complexity and versatility requirements. 

Besides, this thesis follows a stochastic approach for characterizing the variations in the system. In particular, these variations cause uncertainty in the interference power received at the primary system, which may completely disrupt the operation of the CR systems. In order to maintain this uncertainty below a desired level, new interference constraints are proposed in the thesis. Moreover, the theoretical expressions, derived for the performance evaluation, are verified by means of simulations. 


%specially for the channels between the primary and secondary systems. 
%In this context, the proposed framework captures the performance degradation, due to time allocated for performing channel estimation and the imperfect knowledge, the performance with the baseline models, constituting perfect channel knowledge. 


Second, the thesis features performance tradeoffs that determine the maximum achievable throughput of the CR systems while satisfying the interference constraint. At the system design, these tradeoffs provide insights for evaluating the performance degradation in terms of throughput caused due to an inappropriate selection of the estimation and sensing durations. 

Third, using a software defined radio platform, a hardware implementation is carried out to validate the feasibility of the analysis proposed. In addition, a hardware demonstrator is deployed, which in a way presents the operation of a CR system in more practical conditions. %Thereby promoting the evaluation of CR systems. 



\cleardoublepage
\chapter*{Zusammenfassung}
\begin{otherlanguage}{german}
W\"ahrend des letzten Jahrzehnts erf\"ahrt die Drahtloskommunikation ein \"uberaus gro\ss es Wachstum des Datenverkehrs aufgrund der seit jeher wachsenden Anzahl an angeschlossenen Funkger\"aten. Es ist offensichtlich, dass die heutigen Standards zuk\"unftig nicht mehr dazu in der Lage sein werden, das erhebliche Verkehrsaufkommen, welches von diesen Ger\"aten erzeugt wird, zu bew\"altigen. Ein Gro\ss teil dieser Anforderung k\"onnte durch die Nutzung zus\"atzlichen Spektrums erf\"ullt werden. Aufgrund der exklusiven Nutzungsrechte f\"ur das Spektrum unterhalb von \SI{6}{GHz}, ist dieses nicht dazu in der Lage, der Nachfrage nach zus\"atzlichem Spektrum gerecht zu werden, was zu seiner Knappheit f\"uhrt. Daher wird Cognitive Radio (CR) neben Millimeterwellen-Technologie und Visible Light Communications als alternative Spektrumsquelle betrachtet. Die letztgenannten Techniken eignen sich ausschlie\ss lich f\"ur Punkt-zu-Punkt-Kommunikation, was die Mobilit\"at einschr\"ankt. Im Gegensatz dazu ist es das Ziel eines CR Systems, eine effiziente und f\"ur Mobilfunk geeignete Nutzung des Spektrums unterhalb von \SI{6}{GHz} zu erm\"oglichen. Dies wird dadurch erreicht, dass ein sekund\"arer Zugriff auf das lizenzierte Spektrum bei gleichzeitigem Schutz des Lizenznutzers, welcher auch als Prim\"arsystem bezeichnet wird, erfolgt. 

Obwohl sich eine Vielzahl an Werken in der Literatur mit dem Feld von CR besch\"aftigt, wurde die Analyse der Leistungsf\"ahigkeit bislang nicht ausreichend von einem Standpunkt seitens eines Feldeinsatzes beleuchtet. Dies erschwert das Verst\"andnis des Schadensausma\ss es, welches das Prim\"arsystem erf\"ahrt. Aus diesem Grund verfolgt diese Arbeit einen Ansatz zur Analyse der Leistungsf\"ahigkeit eines CR Systems mit Fokus auf den Feldeinsatz dieser Technik. Die Verfolgung dieses Ansatzes hat aufgedeckt, dass die Kenntnis der involvierten Kan\"ale am sekund\"aren Sender grundlegend f\"ur die Realisierung von kognitiven Techniken ist. Jedoch wurde der Aspekt der Kanalkenntnis im Kontext von CR Systemen und inbesondere ihre Auswirkung auf die Leistungsf\"ahigkeit noch nicht genau verstanden. Mit dem Ziel, diese L\"ucke zu schlie\ss en, wird in dieser Arbeit vorgeschlagen, dieses Wissen bei verschiedenen CR Systemen (Interweave, Underlay und hybriden Systemen) gewinnbringend einzusetzen, indem eine Kanalsch\"atzung durchgef\"uhrt wird. Konkret werden in dieser Arbeit die folgenden drei Aspekte, welche mit den zuvorgenannten CR Sysmtemen korrespondieren und verschiedene kognitive Techniken einsetzen, dargelegt: Spectrum-Sensing, Power-Control und deren Kombination.

Um Effekte wie zeitliche Allokation und Schwankungen zu charakterisieren, welche aufgrund der Ber\"ucksichtigung imperfekter Kanalkenntnis enstehen, und die Leistungsf\"ahigkeit des CR Systems beeintr\"achtigen, wird in dieser Arbeit zun\"achst ein analytischer Rahmen aufgestellt. Um den Feldeinsatz eines CR Systems seitens der Hardware zu vereinfachen, wird eine neue Technik namens Received-Power-Based-Estimation zur Sch\"atzung der Kan\"ale  zwischen dem Prim\"ar- und dem Sekund\"arsystem eingesetzt, wodurch Anforderungen hinsichtlich niedriger Komplexit\"at sowie vielf\"altiger Einsetzbarkeit erf\"ullt werden. 

Au\ss erdem wird in dieser Arbeit ein stochastischer Ansatz zur Charakterisierung der Schwankungen im System verfolgt. Inbesondere f\"uhren diese Schwankungen zu Unsicherheit \"uber die Interferenzleistung, welche das Prim\"arsystem erf\"ahrt, wodurch der Betrieb des CR Systems vollst\"andig unterbrochen werden kann. Um die Unsicherheit unter einem gew\"unschten Wert zu halten werden in der Arbeit neuartige Beschr\"ankungen hinsichtlich der Interferenzleistung vorgeschlagen. Dar\"uber hinaus werden die hergeleiteten theoretischen Ausdr\"ucke zur Bewertung der Leistungf\"ahigkeit durch Simulation best\"atigt. 

Als zweiten zentralen Punkt zeigt diese Arbeit Performance-Tradeoffs auf, welche f\"ur den maximal erreichbaren Durchsatz der betrachteten CR Systeme bei gleichzeitiger Einhaltung der Interferenzleistungs-Beschr\"ankung entscheidend sind. F\"ur den Systementwurf liefern diese Zusammenh\"ange wertvolle Einsichten, um die Verschlechterung der Leistungsf\"ahigkeit hinsichtlich des Durchsatzes zu bewerten, welche durch eine unangemessene Wahl der Sch\"atz- und Sensing-Dauern hervorgerufen wird.

Zuletzt wird eine Hardware-Implementierung auf einer Software Defined Radio Plattform durchgef\"urt, um die Realisierbarkeit der vorgestellten Analyse zu bewerten. Zus\"atzlich wird ein Hardware-Vorf\"uhrger\"at aufgebaut, welches in einer gewissen Art und Weise den Betrieb eines CR Systems unter praxisn\"aheren Umst\"anden darstellt.
\end{otherlanguage}
\cleardoublepage
