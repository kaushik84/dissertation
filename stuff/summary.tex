
\chapter*{Abstract}
Over the last decade, wireless communication is witnessing a tremendous growth in the mobile traffic due to ever-increasing number of connected devices. Certainly, the state-of-the-art standards are incapable of accommodating the huge demand of mobile data, originating from these devices. A substantial portion of this demand can be satisfied through an additional spectrum. Due to exclusive usage, the spectrum below $\SI{6}{GHz}$ -- suitable for mobile communications -- is incapable to meet up the demand of the additional spectrum, leading to its scarcity. To this end, cognitive radio, along with millimeter technology and visible-light communication, is envisaged as the powerful source of spectrum. A cognitive radio aims at an efficient utilization of the spectrum below $\SI{6}{GHz}$ by enabling a secondary access to the licensed spectrum while guaranteeing sufficient protection to the licensed users. In this regard, the performance analysis has been critical to a cognitive radio system that, in-turn, encourages its hardware deployment. 


Despite the fact that an extensive amount of literature dedicated to the field of cognitive radio communication, its performance analysis has been dealt inadequately from a deployment perspective. Motivated by this fact, this thesis establishes a deployment-centric viewpoint while analyzing the performance of a cognitive radio system. In this context, it has been emphasized that the channel knowledge is pivotal to the deployment of a cognitive radio system. In this context, the encourages the techniques for the that are easily used for performance analysis and   

\cleardoublepage
\chapter*{Zusammenfassung}


\cleardoublepage
