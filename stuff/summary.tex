
\chapter*{Abstract}
Over the last decade, wireless communication is witnessing a tremendous growth in the mobile traffic due to ever-increasing number of connected devices. Certainly, the state-of-the-art standards are incapable of sustaining the substantial amount of mobile traffic, originating from these devices. It has been identified that a major portion of this requirement can be satisfied through an additional spectrum. Due to exclusive usage, the spectrum below $\SI{6}{GHz}$ -- mainly suitable for mobile communications -- is incapable to meet up the demand of spectrum, leading to its scarcity. To this end, Cognitive Radio (CR), along with millimeter technology and visible-light communication, is envisaged as an alternative source of spectrum. A CR aims at an efficient utilization of the spectrum below $\SI{6}{GHz}$ by enabling a secondary access to the licensed spectrum while guaranteeing sufficient protection to the licensed users (referred as primary system). %Due to this strict constraint on the interference caused by the shared access, the performance analysis has been critical to a cognitive radio system that, in-turn, encourages its hardware deployment. 


Despite the fact that an extensive amount of literature has been dedicated to the field of CR radio communication, its performance analysis has been dealt inadequately from a deployment perspective. Thus, making it difficult to understand the extent of vulnerability, caused in the system, because of its deployment. Motivated by this fact, this thesis establishes a deployment-centric viewpoint for analyzing the performance of a CR system. Following the analysis, it has been identified that the interacting channels' knowledge is pivotal for the realization of cognitive techniques, including spectrum sensing and power control, at the secondary transmitter. The aspect of channels' knowledge, particularly its effect on the performance in context to CR systems has not been clearly understood. This thesis proposes a successful integration of this knowledge -- by carrying out channel estimation -- in reference to different CR systems, namely interweave, underlay and hybrid systems. 

As a major contribution, this thesis establishes an analytical framework to evaluate the harmful effects such as time allocation and variation, arising due to the incorporation of imperfect channel knowledge, on the performance of the CR systems. In order to facilitate hardware deployment, received power-based estimation, a novel channel estimation technique that attributes low-complexity and versatility requirements is employed. 

A probabilistic approach is followed for characterizing the variations in the system. In particular, these variation causes uncertainty in the interference at the primary system, disrupting the operation of CR systems. In this regard, new interference constraints are proposed so that this uncertainty is maintained below a desired level. The theoretical expressions derived for the performance evaluations are verified by means of simulations. 


%specially for the channels between the primary and secondary systems. 
%In this context, the proposed framework captures the performance degradation, due to time allocated for performing channel estimation and the imperfect knowledge, the performance with the baseline models, constituting perfect channel knowledge. 


Second, the thesis characterizes the performance tradeoffs to determine the suitable estimation and suitable sensing durations that attains the maximum performance of the CR systems in terms of the achievable throughput while satisfying the interference constraint.  

Third, a hardware implementation is carried out to validate the feasibility of the analysis proposed and the approach followed in the thesis.  


\cleardoublepage
\chapter*{Zusammenfassung}


\cleardoublepage
