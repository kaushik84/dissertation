
\chapter*{Abstract}
Over the last decade, wireless communication is witnessing a tremendous growth in the mobile traffic due to ever-increasing number of connected devices. Certainly, state-of-the-art standards are incapable of sustaining the substantial amount of mobile traffic, originating from these devices. It is being visualized that a major portion of this requirement can be satisfied through an additional spectrum. Due to exclusive usage, the spectrum below $\SI{6}{GHz}$ is not able to meet up the demand of additional spectrum, leading to its scarcity. To this end, Cognitive Radio (CR), along with millimeter-wave technology and visible-light communication, is envisaged as an alternative source of spectrum. The latter techniques are limited to a point-to-point communication, thus compromising mobility. In contrast, a CR system aims at an efficient utilization of the spectrum below $\SI{6}{GHz}$ -- suitable for mobile communications -- by enabling a secondary access to the licensed spectrum while guaranteeing a sufficient protection to the licensed users (also referred as a primary system). %Due to this strict constraint on the interference caused by the shared access, the performance analysis has been critical to a cognitive radio system that, in-turn, encourages its hardware deployment. 


Despite the fact that an extensive amount of literature has been dedicated to the field of CR communication, its performance analysis has been dealt inadequately from a deployment perspective. Thus, making it difficult to understand the extent of vulnerability, caused to the primary system, because of its deployment. Motivated by this fact, this thesis establishes a deployment-centric viewpoint for analyzing the performance of a CR system. Following this viewpoint, it has been identified that the interacting channels' knowledge is pivotal for the realization of cognitive techniques at the secondary transmitter. However, the aspect of channels' knowledge, particularly its effect on the performance in context to CR systems has not been clearly understood. This thesis proposes a successful integration of this knowledge -- by carrying out channel estimation -- in reference to different CR systems, namely interweave, underlay and hybrid systems. With regard to the employment of the cognitive techniques (that mainly include spectrum sensing, power control and their combination) corresponding to the aforementioned CR systems, the thesis underlines the following three aspects. 

First, this thesis establishes an analytical framework to characterize the effects such as time allocation and variation, arising due to the incorporation of imperfect channel knowledge, that are detrimental to the performance of the CR systems. In order to facilitate hardware deployment of a CR system, received power-based estimation, a novel channel estimation technique is employed for the channels existing between the primary and the secondary systems, thus fulfilling low-complexity and versatility requirements. 

Besides, this thesis follows a probabilistic approach for characterizing the variations in the system. In particular, these variation causes uncertainty in the interference at the primary system, which may disrupt the operation of the CR systems. In order to maintain the uncertainty below a desired level, new interference constraints are proposed in the thesis. Moreover, the theoretical expressions derived for the performance evaluation of the CR systems are verified by means of simulations. 


%specially for the channels between the primary and secondary systems. 
%In this context, the proposed framework captures the performance degradation, due to time allocated for performing channel estimation and the imperfect knowledge, the performance with the baseline models, constituting perfect channel knowledge. 


Second, the thesis features performance tradeoffs that determine the maximum performance of the CR systems in terms of the achievable throughput while satisfying the interference constraint. At the system design, these tradeoffs provide insights for evaluating the performance degradation in terms of throughput caused due to an inappropriate selection of the estimation and sensing durations. 

Third, using a software defined radio platform, a hardware implementation is carried out to validate the feasibility of the analysis proposed as-well-as the approach followed in the thesis. In addition, a hardware demonstrator is deployed that presents the operation of a CR system in more practical conditions. %Thereby promoting the evaluation of CR systems. 



\cleardoublepage
\chapter*{Zusammenfassung}


\cleardoublepage
