%% start of file `template.tex'.
%% Copyright 2006-2013 Xavier Danaux (xdanaux@gmail.com).
%
% This work may be distributed and/or modified under the
% conditions of the LaTeX Project Public License version 1.3c,
% available at http://www.latex-project.org/lppl/.


\documentclass[11pt,a4paper,sans]{moderncv}        % possible options include font size ('10pt', '11pt' and '12pt'), paper size ('a4paper', 'letterpaper', 'a5paper', 'legalpaper', 'executivepaper' and 'landscape') and font family ('sans' and 'roman')

% moderncv themes
\moderncvstyle{casual}                             % style options are 'casual' (default), 'classic', 'oldstyle' and 'banking'
\moderncvcolor{blue}                               % color options 'blue' (default), 'orange', 'green', 'red', 'purple', 'grey' and 'black'
%\renewcommand{\familydefault}{\sfdefault}         % to set the default font; use '\sfdefault' for the default sans serif font, '\rmdefault' for the default roman one, or any tex font name
%\nopagenumbers{}                                  % uncomment to suppress automatic page numbering for CVs longer than one page

% character encoding
\usepackage[utf8]{inputenc}                       % if you are not using xelatex ou lualatex, replace by the encoding you are using
%\usepackage{CJKutf8}                              % if you need to use CJK to typeset your resume in Chinese, Japanese or Korean

% adjust the page margins
\usepackage[scale=0.75]{geometry}
%\setlength{\hintscolumnwidth}{3cm}                % if you want to change the width of the column with the dates
%\setlength{\makecvtitlenamewidth}{10cm}           % for the 'classic' style, if you want to force the width allocated to your name and avoid line breaks. be careful though, the length is normally calculated to avoid any overlap with your personal info; use this at your own typographical risks...

% personal data
\name{Ankit}{Kaushik}
\title{5G Networks Research Scientist Senior Researcher}                               % optional, remove / comment the line if not wanted
\address{Luise Regger Strasse 58}{76137 Karlsruhe}{Germany}% optional, remove / comment the line if not wanted; the "postcode city" and and "country" arguments can be omitted or provided empty
\phone[mobile]{+49~(176)~236~457~33}                   % optional, remove / comment the line if not wanted
\phone[fixed]{+49~(721)~608~437~48}                    % optional, remove / comment the line if not wanted
%\phone[fax]{+3~(456)~789~012}                      % optional, remove / comment the line if not wanted
\email{ankit.kaushik1984@gmail.com}                               % optional, remove / comment the line if not wanted
%\homepage{\url{http://www.cel.kit.edu/english/team_1316.php}}                         % optional, remove / comment the line if not wanted
%\extrainfo{additional information}                 % optional, remove / comment the line if not wanted
\photo[64pt][0.4pt]{picture}                       % optional, remove / comment the line if not wanted; '64pt' is the height the picture must be resized to, 0.4pt is the thickness of the frame around it (put it to 0pt for no frame) and 'picture' is the name of the picture file
\quote{Some quote}                                 % optional, remove / comment the line if not wanted

% to show numerical labels in the bibliography (default is to show no labels); only useful if you make citations in your resume
%\makeatletter
%\renewcommand*{\bibliographyitemlabel}{\@biblabel{\arabic{enumiv}}}
%\makeatother
%\renewcommand*{\bibliographyitemlabel}{[\arabic{enumiv}]}% CONSIDER REPLACING THE ABOVE BY THIS

% bibliography with mutiple entries
%\usepackage{multibib}
%\newcites{book,misc}{{Books},{Others}}
%----------------------------------------------------------------------------------
%            content
%----------------------------------------------------------------------------------
\begin{document}
%-----       letter       ---------------------------------------------------------
% recipient data
\recipient{Recruitment team}{NEC Europe Ltd.\\Kurf\"ursten-Anlage 36, 69115 \\Heidelberg, Germany}
\jobref{1609-216-5GN -- 5G Networks Research Scientist Senior Researcher}
\date{26 October 2016}
\opening{Dear Sir or Madam,}
\closing{Yours faithfully,}
\enclosure[Attached]{curriculum vitae, which includes the list of my publications}          % use an optional argument to use a string other than "Enclosure", or redefine \enclname
\makelettertitle
I will be graduating with a Doctor's degree (Dr.Ing.) in Electrical Engineering in January 2017 from Karlsruhe Institute of Technology, Karlsruhe, Germany. My doctoral thesis primarily focuses on Cognitive Radio that presents an alternative approach of resolving spectrum scarcity for the upcoming 5G networks, by opportunistically accessing the licensed spectrum. As a part my research, I have been able justify my theoretical findings by means of a hardware implementation (a software defined platform). Besides research, I successfully participated in a BMBF (Germen Federal Ministry of Research and Education) project, supervised Master/Bachelor thesis, held tutorials and laboratory during the semester. 

Since I am highly interested in research for 5G networks, I believe starting as a research scientist, offered by NEC, would be next logical step for me. I am approaching you in this matter because your company is working on various aspects like system modeling, performance evaluation and hardware prototyping, which are well related to my research experience. Besides, I am attracted towards the fact that NEC provides a great opportunity for presenting the innovative ideas in the form of publications and patents. Despite the fact that my research experience has been confined mostly to the physical layer aspects, I am willing extend my knowledge to computer networks, particularly in the direction of software defined networking and the related tools.   

I would like to contribute to the ongoing research at NEC and hope that my profile fits to your workgroup. I am looking forward to hearing form you. 


\makeletterclosing

\end{document}


%% end of file `template.tex'.
