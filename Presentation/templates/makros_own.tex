%%%%%%%%%%%%%%%%%%%%%%%%
% eigene Kommandes
%

\usepackage{trfsigns}

% transform as symbol
\def\transform{\; \laplace \;}
\def\Transform{\; \Laplace \;}

\newcommand{\vTransform}[1][]{\mbox{\setlength{\unitlength}{0.1em}%
		\begin{picture}(10,20)%
		\put(3,2){\circle{4}}%
		\put(3,4){\line(0,1){12}}%
		\put(3,18){\circle*{4}}%
		\put(10,7){#1}
		\end{picture}%
	}%
}%

\newcommand{\vtransform}[1][]{\mbox{\setlength{\unitlength}{0.1em}%
		\begin{picture}(10,20)%
		\put(3,2){\circle*{4}}%
		\put(3,4){\line(0,1){12}}%
		\put(3,18){\circle{4}}%
		\put(10,7){#1}
		\end{picture}%
	}%
}%     


% Schlagwort
\newcommand\schlagwort[1]{\textbf{\textcolor{kit-green100}{#1}}}

% Farben f�r Boxen
\setbeamercolor{color_lower}{fg=black,bg=kit-green15}
\setbeamercolor{color_upper}{fg=black,bg=kit-green70}
%\footnotesize\tiny

%Zahlenk�rper
\def\rz{\ifmmode{\mathbb{R}}%
    \else{\hbox{$\mathbb{R}$}}\fi} 
\def\nz{\ifmmode{\mathbb{N}}%
    \else{\hbox{$\mathbb{N}$}}\fi} 
\def\gz{\ifmmode{\mathbb{Z}}%
   \else{\hbox{$\mathbb{Z}$}}\fi} 
\def\cz{\ifmmode{\mathbb{C}}
    \else{\hbox{$\mathbb{C}$}}\fi}%
\def\qz{\ifmmode{\mathbb{Q}}%
    \else{\hbox{$\mathbb{Q}$}}\fi}%   
\def\K{\ifmmode{\mathbb{K}}%
    \else{\hbox{$\mathbb{K}$}}\fi}%  

% Erwartungswert
\def\Er{\ifmmode{\mathbb{E}}%
    \else{\hbox{$\mathbb{E}$}}\fi}%  

% Real- und Imagin�rteil
\def\real{{\text{Re}}}
\def\imag{{\text{Im}}}

% simple implication within the text
%\def\thus{{$\implies$}}
\def\thus{\relax
	\ifmmode
		\implies
	\else
		$\implies$
	\fi}

%Mengen durch kaligraphische Buchstaben
\def\setS{\mathcal{S}}
\def\setP{\mathcal{P}}
\def\setX{\mathcal{X}}
\def\setY{\mathcal{Y}}
\def\setZ{\mathcal{Z}}
\def\setC{\mathcal{C}}
\def\defl{:=}
\def\Pr{P}

% Befehl zur Ausrichtung der itemize-Bullets in Columns; erfordert in den Frames ein \AdjustMargins
\makeatletter
\newcommand*{\AdjustMargins}{%
    \setlength{\beamer@rightmargin}{0em}%
    \setlength{\beamer@leftmargin}{0em}%
}
\makeatother

% Declare operators
\DeclareMathOperator*{\mini}{min}
\DeclareMathOperator*{\argmin}{arg\,min}
\DeclareMathOperator*{\maxi}{max}


% colored small box representing end of a proof
\def\endofproof{
	\ifmmode
		\text{\hfill} \textcolor{KITgreen}{\rule{2mm}{2mm}}
	\else
		$\hfill \textcolor{KITgreen}{\rule{2mm}{2mm}}$
	\fi}

% colored box for definition and theorems
\newcommand\theobox[2]{
	\begin{center}
	\begin{beamerboxesrounded}
	[upper=color_upper,lower=color_lower,shadow=true]
	{\textcolor{white}{\textbf{#1}}}
	#2
	\end{beamerboxesrounded}
	\end{center}
}


% Boxed equation 
\newcommand{\eqbox}[1]{
\begin{center}
\setlength{\fboxsep}{2mm}
\setlength{\fboxrule}{0.3mm}
\fcolorbox{kit-green100}{white}
%{\parbox{.9\linewidth \fboxsep \fboxrule}{
{\parbox{.9\linewidth }{
\bigskip
#1
\bigskip
}}
\end{center}
}


% new definition of for a boxed equation 
\newcommand{\eqboxed}[1]{
\begin{center}
\fcolorbox{kit-green100}{white}{
$#1$
}
\end{center}}


% Literature as intended
\setbeamertemplate{bibliography item}[text]


% Befehl zur Erh�hung der Section number
\makeatletter
\newcommand{\setnextsection}[1]{%
	\setcounter{section}{\numexpr#1-1\relax}%
	\beamer@tocsectionnumber=\numexpr#1-1\relax\space}
\makeatother

	
% Direktes Fu�note bauen und zitieren
\def\footcit#1{\footnote{\scriptsize Nach \cite{#1}}}


% Konstruktion einer �bersicht bei jedem Aufruf von \subsection
% Nur die aktuelle ist schwarz, alle anderen sind grau
\AtBeginSubsection[]
{
  \begin{frame}[t] 
       \frametitle{�bersicht}
       \tableofcontents[currentsection,currentsubsection]
  \end{frame}
}



%%%%%%%%%%%%%%%%%%%%%%%%%%%%%%%%%%%%%%%%%%%%%%%%%%%%%%
% Private "`Testsektion"'
\setbeamercolor{color_lower}{fg=KITblack,bg=kit-green15}
\setbeamercolor{color_upper}{fg=KITblack,bg=kit-green70}
\setbeamertemplate{navigation symbols}{}

\def\colorize<#1>{%
	\temporal<#1>{\color{KITblack}}{\color{KITblack30}}{\color{KITblack}}}

\definecolor{KITgreen}{rgb}{0,.59,.51}
\definecolor{KITblack}{rgb}{0,0,0}


% Fu�note anpassen, so dass es nicht mit dem Footer �berschneidet
\addtobeamertemplate{footnote}{\vspace{-6pt}\advance\hsize-0.5cm}{\vspace{6pt}}
\renewcommand*{\footnoterule}{\kern -3pt \hrule width 1in \kern 8.6pt}
\footnotesize\scriptsize



\newcommand\FourQuad[4]{%
	\begin{minipage}[b][.35\textheight][t]{.47\textwidth}#1\end{minipage}\hfill%
	\begin{minipage}[b][.35\textheight][t]{.47\textwidth}#2\end{minipage}\\[1em]
	\begin{minipage}[b][.35\textheight][t]{.47\textwidth}#3\end{minipage}\hfill
	\begin{minipage}[b][.35\textheight][t]{.47\textwidth}#4\end{minipage}%
}

%%%%%%%%%%%%%%%%%%%%%%%%%%%%%%%%%%%%%%%%%%%%%%%%%%%%%%%%%%%%%%%%%%%%%
%% Code to spread out slides contents using \stretchy command: (inserted by Jo)
\let\svpar\par
\let\svitemize\itemize
\let\svenditemize\enditemize
\let\svitem\item
\def\newpar{\def\par{\svpar\vfill}}
\def\newitem{\def\item{\vfill\svitem}}
\let\svcenter\center
\let\svendcenter\endcenter
\let\svcolumn\column
\let\svendcolumn\endcolumn
\newlength\columnskip
\columnskip 0pt
\def\newcolumn{%
	\renewenvironment{column}[2]%
	{\svcolumn{##1}\setlength{\parskip}{\columnskip}##2}%
	{\svendcolumn\vspace{\columnskip}}}

\newcommand\stretchy{\only<1>{%
		\newpar\def\item{\svitem\newitem}%
		\renewenvironment{itemize}{\svitemize}{\svenditemize\newpar\par}%
		\renewenvironment{center}{\svcenter\newpar}{\svendcenter\newpar}%
		\newcolumn
	}}