
\chapter{Acronyms and Abbreviations}
\renewcommand{\arraystretch}{1.4}
\begin{longtable}{p{0.2\columnwidth}p{0.8\columnwidth}}
%  \renewcommand{\arraystretch}{1.4}
%  \begin{tabular}{p{0.3\columnwidth}p{0.7\columnwidth}}
   	AC 	&	Average Constraint\\
        CSC 	&	Cognitive Small Cell\\
	CSC-BS	&	Cognitive Small Cell-Base Station \\
%% \acro{CFAR}{Constant False Alarm Rate}
  	EM	&	Estimation Model (Proposed Approach) \\
%% \acro{EM}{Expectation and Maximization}
	HS	& 	Hybrid System \\
	ID	&	Indoor Device \\
%% \acro{IC}{Interference Constraint}
	IS	& 	Interweave System \\
%% \acro{MLE}{Maximum Likelihood Estimation}
%% \acro{MM}{Mixture Model}
	OS	&	Overlay System \\
	OC	&	Outage Constraint \\
	PR	& 	Primary Receiver \\
	PU	& 	Primary User \\
	PT	& 	Primary Transmitter \\
%% \acro{ROC}{Receiver Operating Characteristics}
	SNR	&	Signal to Noise Ratio \\
	SR	& 	Secondary Receiver \\
	SU	&	Secondary User \\
	ST	& 	Secondary Transmitter \\
	US	&	Underlay System \\

%% Abbreviations  
	cdf	&       cumulative distribution function \\
	cf.	&	confer (refer to) \\ 
 	e.g.	& 	exempli gratia (for example) \\
 	i.e.	& 	id est (that is) \\ 
	pdf	&       probability density function \\
 	Rx	& 	Receiver \\
 	Tx	& 	Transmitter

 % \end{tabular}
\end{longtable}
  


%\begin{acronym}[ANNNNOVA]
%  \acro{AC}{Average Constraint}
%% \acro{CC}{Capacity Constraint}
%% \acro{CR}{Cognitive Relay}
% \acro{CSC}{Cognitive Small Cell}
% \acro{CSC-BS}{Cognitive Small Cell-Base Station}
%% \acro{CFAR}{Constant False Alarm Rate}
%  \acro{EM}{Estimation Model (Proposed Approach)}
%% \acro{EM}{Expectation and Maximization}
% \acro{ID}{Indoor Device}
%% \acro{IC}{Interference Constraint}
% \acro{IS}{Interweave System}
%% \acro{MLE}{Maximum Likelihood Estimation}
%% \acro{MM}{Mixture Model}
% \acro{OS}{Overlay System}
% \acro{OC}{Outage Constraint}
% \acro{PR}{Primary Receiver}
% \acro{PU}{Primary User}
% \acro{PT}{Primary Transmitter}
%% \acro{ROC}{Receiver Operating Characteristics}
% \acro{SNR}{Signal to Noise Ratio}
% \acro{SR}{Secondary Receiver}
% \acro{SU}{Secondary User}
% \acro{ST}{Secondary Transmitter}
% \acro{US}{Underlay System}
%\end{acronym}

%\section*{Abbreviations}
%\begin{table*}
%  \renewcommand{\arraystretch}{1.4}
%  \begin{tabular}{p{0.3\columnwidth}p{0.7\columnwidth}}
%   \end{tabular}



%\end{table*}
%
%\section*{Notations}
%\begin{table*}
%  \renewcommand{\arraystretch}{1.4}
%  \begin{tabular}{p{0.3\columnwidth}p{0.7\columnwidth}}
% 	$T$			& 	Frame Duration \\
% 	$\fsam$			&	Sampling Frequency \\
% 	$\tsen$			&	Sensing Time \\
% 	$\test$			&	Estimation Time \\	
% 	$\rs$			& 	Secondary Throughput at SR \\
% 	$\cz,\co$		& 	Date rate at SR without and with interference from PT  \\
% 	$\hpo$			&	Channel coefficient for the link PT-ST for IS \\
% 	$\hs$			& 	Channel coefficient for the link ST-SR for IS \\
% 	$\hpt$			& 	Channel coefficient for the link PT-SR for IS \\
% 	$\snrrcvd, \snrso$	&	Signal to noise ratio for the link PT-ST, ST-SR for IS \\
% %\acro{}[$$]{}
% 	$\snrpt$		&	Interference (from PT) to noise ratio for the link PT-SR for IS \\
% 	$\pd$			& 	Detection Probability \\ 
% 	$\pfa$		 	&	False Alarm Probability \\ 
%	$\pdd$			&	Target Detection Probability \\ 
% 	$\mu$			&	Decision Threshold \\ 
%	$\mpd$			&	Outage Constraint over Detection Probability \\ 
%	$F_{(\cdot)}$		&	Cumulative distribution function of random variable $(\cdot)$ \\
%	$f_{(\cdot)}$		&	Probability density function of random variable $(\cdot)$ \\
%	$\mathbb E_{(\cdot)}$	&	Expectation with respect to ($\cdot$) \\
%	$\p$			&	Probability measure \\
%	\textbf{T}$(\cdot)$	&	Test Statistics \\
% 	$\tilde{(\cdot)}$	&	Suitable value of the parameter ($\cdot$) that achieves maximum performance \\
% 	$\tilde{(\cdot)}$	&	Suitable value of the parameter ($\cdot$) that achieves maximum performance \\
% 	$\Ks$			& 	Number of pilot symbols used for pilot based estimation at the SR for $\hs$ \\
% 	$\Kp$			&	Number of samples used for received power based estimation at the SR for $\hpt$ \\
%   \end{tabular}
%\end{table*}
%
%
