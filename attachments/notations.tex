
\chapter{Notations}
\renewcommand{\arraystretch}{1.4}
\begin{longtable}{p{0.3\columnwidth}p{0.7\columnwidth}}

%% System parameters
       $T$                     &       Frame duration \\
       $\fsam$                 &       Sampling frequency \\
       $\flo$                  &       Local oscillator offset frequency \\
       $\tsen$                 &       Sensing time \\
       $\test$                 &       Estimation time \\      
       $\ttesto$               &       Estimation time for the demonstrator \\      
       $\rs$                   &       Secondary throughput at SR \\
       $\cz,\co$               &       Date rate at SR without and with interference from PT  \\
      
%% Channel parameters 
       $\hpo$                  &       Channel coefficient for the link PT-ST for IS \\
       $\hs$                   &       Channel coefficient for the link ST-SR for IS \\
       $\hpt$                  &       Channel coefficient for the link PT-SR for IS \\
       $\snrrcvd, \snrso$      &       Signal to noise ratio for the link PT-ST, ST-SR for IS \\
 %\acro{}[$$]{}
       $\snrpt$                &       Interference (from PT) to noise ratio for the link PT-SR for IS \\

%% Probabilities and constraints
       $\pd$                   &       Detection probability \\ 
       $\pfa$                  &       False alarm probability \\ 
       $\pdd$                  &       Target detection probability \\ 
       $\pco$                  &       Confidence probability \\ 
       $\pcod$                 &       Target confidence probability \\ 
       $\mu$                   &       Decision threshold \\ 
       $\mpd$                  &       Outage constraint over detection probability \\ 
       $\ite$                  &       Interference temperature at the PR \\ 
       $\acc$	               &       Accuracy of the parameter \\	
       $\epsilon$	       &       Relative error between the normalized histogram bins and probability density function  \\	

%% Signals


%% Powers
	$\prcvd$  		&	Power received at ST over the PT-SR link (IS) PR-ST (US) \\ 
	$\prcvdpr$  		&	Power received at PR over the SR-PT link \\

	$\preg$	  		&  	Transmit power at ST, where power control mechanism is enabled	\\
	$\ptranst$ or $\pfull$	 	&  	Transmit power at ST, where power control mechanism is disabled	\\
	
%% Notations 
       $F_{(\cdot)}$           &       Cumulative distribution function of random variable $(\cdot)$ \\
       $f_{(\cdot)}$           &       Probability density function of random variable $(\cdot)$ \\
       $\mathbb E_{(\cdot)}$   &       Expectation with respect to ($\cdot$) \\
       $\p$                    &       Probability measure \\
       T$(\cdot)$  	       &       Test statistics \\
       $\tilde{(\cdot)}$       &       Suitable value of the parameter ($\cdot$) that achieves maximum performance \\
       $\tilde{(\cdot)}$       &       Suitable value of the parameter ($\cdot$) that achieves maximum performance \\
       $K$                     &       Scaling factor that holds the expected power received at the PR at interference temperature  \\
       $\Ks$                   &       Number of pilot symbols used for pilot based estimation at the SR for $\hs$ \\
       $\Kp$                   &       Number of samples used for received power based estimation at the SR for $\hpt$ \\
       $\ncchi2$               &       Non-central chi-squared distribution \\
       $\lambda_{(\cdot)}$ or $\lambda$       &       Non-centrality parameter of a non-central chi-squared distribution \\
       $a_{(\cdot)}, b_{(\cdot)}$ or $a, b$       &       Shape and scale parameters of a Gamma distribution \\

%% Special Functions
       $I_{N}(\cdot)$	       &	Modified Bessel function of first kind of order $N$ \\		
       $Q_{N}(\cdot, \cdot)$	       &	Marcum Q-function \\		
\end{longtable}
  



%\section*{Abbreviations}
%\begin{table*}
%  \renewcommand{\arraystretch}{1.4}
%  \begin{tabular}{p{0.3\columnwidth}p{0.7\columnwidth}}
%	cf.	&	confer (refer to) \\ 
% 	e.g.	& 	exempli gratia (for example) \\
% 	i.e.	& 	id est (that is) \\ 
% 	Rx	& 	Receiver \\
% 	Tx	& 	Transmitter
%   \end{tabular}

