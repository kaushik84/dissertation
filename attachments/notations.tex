
\chapter{Notations and Symbols}

Please note that, there exists certain notations that are specific or follow a slightly different definition corresponding to the underlying cognitive radio system (interweave system (IS), underlay system (US) and hybrid system (HS)). Here, these notations have been signfied by their underlying systems.  

\renewcommand{\arraystretch}{1.4}
\begin{longtable}{p{0.15\columnwidth}p{0.10\columnwidth}p{0.75\columnwidth}}

%% System parameters
       $T$                     & &             Frame duration \\
       $\fsam$                 & &             Sampling frequency \\
       $\flo$                  & & 	       Local oscillator offset frequency \\
       $\tsen$                 & &             Sensing time interval \\
       $\test$                 & &             Estimation time interval \\      
       $\testpt$                 & HS &             Estimation time interval allocated for $\hpo$ \\      
       $\testptsr$                 & HS &             Estimation time interval allocated for $\hptw$ \\      
       $\testpr$                 & HS &             Estimation time interval allocated for $\hpth$ \\      
       $\ttesto$               & &             Estimation time interval for the demonstrator \\      
       $\rs$                   & &             Throughput at SR (secondary throughput) \\
       $\cz$                   & IS, HS &      Date rate at SR without interference from PT, where no power control is employed at ST  \\
       $\co$                   & IS, HS &      Date rate at SR without interference from PT, where no power control is employed at ST \\ 
       $\ca$                   & US &          Date rate at SR, where power control is employed at ST \\
       $\ctw$                   & IS, HS &      Date rate at SR without interference from PT, where power control is employed at ST  \\
       $\cth$                   & IS, HS &      Date rate at SR with interference from PT, where power control is employed at ST  \\ 

%% Channel parameters 
       $\hpo$                  & IS, HS &         Channel coefficient for the link PT-ST \\
       $\hpo$                  & US &             Channel coefficient for the link PR-ST \\
       $\hs$                   &  &     Channel coefficient for the link ST-SR \\
       $\hpt$                  &  &     Channel coefficient for the link PT-SR \\
       $\hpth$                 & HS &             Channel coefficient for the link PR-ST \\
 %\acro{}[$$]{}

%% Probabilities and constraints
       $\pd$                   & &             Detection probability \\ 
       $\pfa$                  & &             False alarm probability \\ 
       $\pco$                  & &             Confidence probability \\ 
       $\pdd$                  & &             Target detection probability \\ 
       $\pcod$                 & &             Target confidence probability \\ 
       $\mu$                   & &             Decision threshold \\ 
       $\mpd$                  & &             Outage constraint over detection probability at ST \\ 
       $\opc$                  & &             Outage constraint on controlled power at ST \\ 
       $\ite$                  & &             Interference temperature at PR \\ 
       $\acc$	               & &             Accuracy of the parameter \\	
       $\epsilon$	       & &             Relative error between the normalized histogram bins and probability density function  \\	

%% Signals
	$\xp[\cdot]$       & &             Discrete and complex signal transmitted by PT \\	
	$\xtranpr[\cdot]$       & &             Discrete and complex signal transmitted by PR \\	
	$\xscont[\cdot]$       & US, HS &             Discrete and complex signal transmitted by ST with controlled power \\	
	$\xs[\cdot]$       & IS, HS &             Discrete and complex signal transmitted by ST with no power control\\	

	$\yrcvd[\cdot]$       & &             Discrete and complex signal received at ST \\	
	
	$\nas[\cdot]$       & &             Discrete and complex noise signal \\	
	%$\nap[\cdot]$       & &             Discrete and complex noise signal at PR \\	

	$\yp[\cdot]$       & &             Discrete and complex signal recived at PR \\	
%% Powers
	$\prcvd$  		& IS &		Power received at ST over the PT-ST \\ 
	$\prcvd$  		& US &		Power received at ST over the PR-ST \\ 
	$\prcvdstpt$  		& HS &		Power received at ST over the PT-ST over $\hpo$ \\ 
	$\prcvdstpr$  		& HS &		Power received at ST over the PR-ST over $\hpth$ \\ 
	$\prcvdpr$  		& US, HS &		Interference power received at PR over the ST-PR link \\
	$\prcvdsr$  		& &		Interference power received at SR over the PT-SR link\\ 

	$\preg$	  		& US &  	Transmit power at ST with power control \\
	$\ptranst$ 		& IS, HS&  	Transmit power at ST with full transmit control \\
	$\ptranpr$  		& &		Transmit power at PR \\

	$\npo$	 		& & 	Noise power \\
	$\spo$	 		& &  Transmit power at ST and SR, when transmit signal is modeled as OFDM  \\

%% Signal to noise ratio
	$\snrrcvd$  		& IS &		Signal to noise power received at ST over PT-ST link \\ 
	$\snrrcvd$  		& US &		Signal to noise power received at ST over PR-ST link \\ 
        $\snrpt$                & &             Interference (from PT) to noise ratio for PT-SR link \\
%% Notations 
       $F_{(\cdot)}$            & &          Cumulative distribution function of random variable $(\cdot)$ \\
       $f_{(\cdot)}$            & &          Probability density function of random variable $(\cdot)$ \\
       $\mathbb E_{(\cdot)}$    & &		Expectation with respect to ($\cdot$) \\
       $\p$                     & &     	Probability measure \\
       T$(\cdot)$  	        & &          Test statistics \\
       $\tilde{(\cdot)}$        & &		Suitable value of the parameter ($\cdot$) that achieves maximum performance \\
       $\tilde{(\cdot)}$        & &		Suitable value of the parameter ($\cdot$) that achieves maximum performance \\
       $K$                      & &      Scaling factor that holds the expected power received at the PR at interference temperature  \\
       $\Ks$                    & &      Number of pilot symbols used for pilot based estimation at the SR \\
       $\Kp$                    & &      Number of samples used for received power based estimation at the SR \\
       $\mathcal N$                & &      Gaussian or normal distribution\\
       $\cchi2 $                & &      Central chi-squared distribution \\
       $\ncchi2$                & &      Non-central chi-squared distribution \\
       $\lambda_{(\cdot)}$ or $\lambda$  &     &       Non-centrality parameter of a non-central chi-squared distribution \\
       $a_{(\cdot)}, b_{(\cdot)}$ or $a, b$ &  &       Shape and scale parameters of a Gamma distribution \\

%% Special Functions
       $I_{N}(\cdot)$	        & &	Modified Bessel function of first kind of order $N$ \\		
       $Q_{N}(\cdot, \cdot)$	& &	Marcum Q-function \\		
       $\Gamma(\cdot, \cdot)$	& &	Regularized upper-incomplete Gamma function\\		
       $\Gamma^{-1}(\cdot, \cdot)$	& &	Inverse function of regularized upper-incomplete Gamma function\\		
       %$\gamma(\cdot, \cdot)$	& &	Regularized lower-incomplete Gamma function or Gamma distribution function\\		
       $\mathcal H_1$			& &	Hypothesis illustarting the prescence of primary user\\	
       $\mathcal H_0$			& &	Hypothesis illustarting the absence of primary user (noise only) \\		
\end{longtable}
  



%\section*{Abbreviations}
%\begin{table*}
%  \renewcommand{\arraystretch}{1.4}
%  \begin{tabular}{p{0.3\columnwidth}p{0.7\columnwidth}}
%	cf.	&	confer (refer to) \\ 
% 	e.g.	& 	exempli gratia (for example) \\
% 	i.e.	& 	id est (that is) \\ 
% 	Rx	& 	Receiver \\
% 	Tx	& 	Transmitter
%   \end{tabular}

